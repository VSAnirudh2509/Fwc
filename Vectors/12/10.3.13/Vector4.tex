\documentclass[12pt]{article}
\usepackage{graphicx}
%\documentclass[journal,12pt,twocolumn]{IEEEtran}
\usepackage[none]{hyphenat}
\usepackage{graphicx}
\usepackage{listings}
\usepackage[english]{babel}
\usepackage{graphicx}
\usepackage{caption} 
\usepackage{hyperref}
\usepackage{booktabs}
\usepackage{commath}
\usepackage{gensymb}
\usepackage{array}
\usepackage{amsmath}   % for having text in math mode
\usepackage{listings}
\let\vec\mathbf
\lstset{
  frame=single,
  breaklines=true
}
  
%Following 2 lines were added to remove the blank page at the beginning
\usepackage{atbegshi}% http://ctan.org/pkg/atbegshi
\AtBeginDocument{\AtBeginShipoutNext{\AtBeginShipoutDiscard}}
%
%New macro definitions
\newcommand{\mydet}[1]{\ensuremath{\begin{vmatrix}#1\end{vmatrix}}}
\providecommand{\brak}[1]{\ensuremath{\left(#1\right)}}
\providecommand{\norm}[1]{\left\lVert#1\right\rVert}
\newcommand{\solution}{\noindent \textbf{Solution: }}
\newcommand{\myvec}[1]{\ensuremath{\begin{pmatrix}#1\end{pmatrix}}}
\let\vec\mathbf
\begin{document}
\begin{center}
\title{\textbf{Vector Algebra}}
\date{\vspace{-5ex}} %Not to print date automatically
\maketitle
\end{center}
\setcounter{page}{1}
\section*{CHAPTER 10 - VECTOR ALGEBRA}
\section*{Excercise 10.3}
\solution 
\begin{enumerate}
\item If $\overrightarrow{a},\overrightarrow{b},\overrightarrow{c}$ are unit vectors such that $\overrightarrow{a}+\overrightarrow{b}+\overrightarrow{c}=0$, find the value of $\overrightarrow{a}.\overrightarrow{b}+\overrightarrow{b}.\overrightarrow{c}+\overrightarrow{c}.\overrightarrow{a}$.  
\section{Solution}
The given vectors $\vec{a},\vec{b}$ and $\vec{c}$ are unit vectors. Since the given vectors $\vec{a},\vec{b},\vec{c}$ are unit vector hence $\vec{a}=\vec{b}=\vec{c}$ which is equal to 1.
        \begin{align}
\norm{\vec{a}} &=\sqrt{1^2}=1\\ \norm{\vec{b}}&=\sqrt{1^2}=1\\ \norm{\vec{c}}&=\sqrt{1^2}=1
        \end{align}
The Given equation is 
        \begin{align}
\vec{a}+\vec{b}+\vec{c}=0
\end{align}      
Squaring on both sides,
\begin{align}
\norm{{\vec{a}}+{\vec{b}}+{\vec{c}}}^2=0^2
\end{align}
\begin{align}
\implies{\norm{\vec{a}}}^2+{\norm{\vec{b}}}^2+{\norm{\vec{c}}}^2+2({{\vec{a}^\top}{\vec{b}}+{\vec{b}^\top}{\vec{c}}+{\vec{c}^\top}{\vec{a}}})=0\\
\implies{\vec{a}^\top\vec{a}}+{\vec{b}^\top\vec{b}}+{\vec{c}^\top\vec{c}}+2({{\vec{a}^\top}{\vec{b}}+{\vec{b}^\top}{\vec{c}}+{\vec{c}^\top}{\vec{a}}})=0\\
\implies{1}^2+{1}^2+{1}^2+2({{\vec{a}^\top}{\vec{b}}+{\vec{b}^\top}{\vec{c}}+{\vec{c}^\top}{\vec{a}}})=0\\
\implies3+2({{\vec{a}^\top}{\vec{b}}+{\vec{b}^\top}{\vec{c}}+{\vec{c}^\top}{\vec{a}}})=0\\
\implies{\vec{a}^\top}{\vec{b}}+{\vec{b}^\top}{\vec{c}}+{\vec{c}^\top}=\frac{-3}{2}
\end{align}
Hence the value of $\overrightarrow{a}.\overrightarrow{b}+\overrightarrow{b}.\overrightarrow{c}+\overrightarrow{c}.\overrightarrow{a}$ is -1.5 or -3/2
\end{enumerate}
\end{document}