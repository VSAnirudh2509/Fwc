\documentclass[12pt]{article}
\usepackage{graphicx}
%\documentclass[journal,12pt,twocolumn]{IEEEtran}
\usepackage[none]{hyphenat}
\usepackage{graphicx}
\usepackage{listings}
\usepackage[english]{babel}
\usepackage{graphicx}
\usepackage{caption}
\usepackage[parfill]{parskip}
\usepackage{hyperref}
\usepackage{booktabs}
%\usepackage{setspace}\doublespacing\pagestyle{plain}
\def\inputGnumericTable{}
\usepackage{color}                                            %%
    \usepackage{array}                                            %%
    \usepackage{longtable}                                        %%
    \usepackage{calc}                                             %%
    \usepackage{multirow}                                         %%
    \usepackage{hhline}                                           %%
    \usepackage{ifthen}
\usepackage{array}
\usepackage{amsmath}   % for having text in math mode
\usepackage{parallel,enumitem}
\usepackage{listings}
\lstset{
language=tex,
frame=single,
breaklines=true
}
 
%Following 2 lines were added to remove the blank page at the beginning
\usepackage{atbegshi}% http://ctan.org/pkg/atbegshi
\AtBeginDocument{\AtBeginShipoutNext{\AtBeginShipoutDiscard}}
%
%New macro definitions
\newcommand{\mydet}[1]{\ensuremath{\begin{vmatrix}#1\end{vmatrix}}}
\providecommand{\brak}[1]{\ensuremath{\left(#1\right)}}
\providecommand{\norm}[1]{\left\lVert#1\right\rVert}
\newcommand{\solution}{\noindent \textbf{Solution: }}
\newcommand{\myvec}[1]{\ensuremath{\begin{pmatrix}#1\end{pmatrix}}}
\let\vec\mathbf
\begin{document}
\begin{center}
\enlargethispage{-4cm}
\title{\textbf{Vector Algebra}}
\date{\vspace{-5ex}} %Not to print date automatically
\maketitle
\end{center}
\setcounter{page}{1}
\section*{CHAPTER 10 - VECTOR ALGEBRA}
\section*{Excercise 10.2}
\solution 
\begin{enumerate}
\item Find the direction cosines of the vector joining the points A (1, 2, –3) and B(–1, –2, 1), directed from $\vec{A}$ and $\vec{B}$.The direction cosines are the cosines of the angles formed by the given vector with the respective axes, given vectors are $\vec{A}$ and $\vec{B}$.
\begin{align}
	\vec{A} =\myvec{1\\2\\-3} , \vec{B}=\myvec{-1\\-2\\1}
\end{align}
The direction vector m of the line joining two points A, B is given by
\begin{align}
	\vec{B-A} = \myvec{-1\\-2\\1}-\myvec{1\\2\\-3}=\myvec{-2\\-4\\4}
\end{align}
\begin{align}
	\vec{m}=\vec{A-B}=\myvec{-2\\-4\\4}
\end{align}
\begin{align}
	\norm{\vec{m}}=\sqrt{(-2)^2+(-4)^2+4^2}=6
\end{align}
The unit vector is given by 
\begin{align}
\hat{\vec{m}}=\frac{\vec{m}}{\norm{\vec{m}}}
\end{align}
Hence, the unit vector in the direction of m is calculated as
\begin{align}
	\frac{\vec{m}}{\norm{\vec{m}}}=\frac{1}{6}{\myvec{-2\\-4\\4}}=\myvec{\frac{-1}{3}\\[4pt] \frac{-2}{3}\\[4pt] \frac{2}{3}}
\end{align}
Hence, the  direction cosine of vector joining points $\vec{A}$ and $\vec{B}$ is,
\begin{align}
\hat{\vec{m}}=\myvec{\frac{-1}{3}\\[4pt] \frac{-2}{3}\\[4pt] \frac{2}{3}}
\end{align}
\end{enumerate}
\end{document}
