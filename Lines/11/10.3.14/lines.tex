\documentclass[12pt]{article}
\usepackage{graphicx}
%\documentclass[journal,12pt,twocolumn]{IEEEtran}
\usepackage[none]{hyphenat}
\usepackage{graphicx}
\usepackage{listings}
\usepackage[english]{babel}
\usepackage{graphicx}
\usepackage{caption} 
\usepackage{hyperref}
\usepackage{booktabs}
\usepackage{commath}
\usepackage{gensymb}
\usepackage{array}
\usepackage{amsmath}   % for having text in math mode
\usepackage{listings}
\let\vec\mathbf
\lstset{
  frame=single,
  breaklines=true
}
  
%Following 2 lines were added to remove the blank page at the beginning
\usepackage{atbegshi}% http://ctan.org/pkg/atbegshi
\AtBeginDocument{\AtBeginShipoutNext{\AtBeginShipoutDiscard}}
%
%New macro definitions
\newcommand{\mydet}[1]{\ensuremath{\begin{vmatrix}#1\end{vmatrix}}}
\providecommand{\brak}[1]{\ensuremath{\left(#1\right)}}
\providecommand{\norm}[1]{\left\lVert#1\right\rVert}
\newcommand{\solution}{\noindent \textbf{Solution: }}
\newcommand{\myvec}[1]{\ensuremath{\begin{pmatrix}#1\end{pmatrix}}}
\let\vec\mathbf
\begin{document}
\begin{center}
\title{\textbf{LINES}}
\date{\vspace{-5ex}} %Not to print date automatically
\maketitle
\end{center}
\setcounter{page}{1}
\section*{CHAPTER 11 - STRAIGHT LINES}
\section*{Excercise 10.3}
\solution 
\begin{enumerate}
\item[Q14.] Find the coordinates of the foot of the perpendicular from (-1 3) to the line 3x-4y-16=0.  
\section{Solution}
The given perpendicular point can be taken as $\vec{P(-1,3)}$.Let us assume the coordinates.
\begin{align}
\vec{A(a,b)}
\end{align}
To find the coordinates of foot of perpendicular from point to line is given as:
\begin{align}
\myvec{
m&n
}^\top A = \myvec{
m^\top\\
c
}
\end{align}
Now, line given 
\begin{align}
3x-4y-16 &=0\\
3x-4y&=16
\end{align}
Therefore, it can be equated as
\begin{align}
\vec{n}^\top \vec{x} = \vec{c}      \label{4}
\end{align}
where,
\begin{align}
\vec{n}=\myvec{
3\\
-4
}, \vec{c}=16
\end{align}
Here $\vec{m}$ is directional vector of the given line
\begin{align}
\vec{m}=\myvec{
4\\3}
\end{align}
Substituting all values in (4), we get
\begin{align}
\myvec
{4&3\\3&-4}A&=\myvec{(4&3)&\myvec{-1\\3}\\&16}\\
\myvec
{4&3\\3&-4}A&=\myvec{5\\16}  \label{9}
\end{align}
The augmented matrix for the system equations in (9) is expressed as
\begin{align}
  \begin{pmatrix}
  \begin{array}{rr|r}
   4 &  3  & 5\\
   3 & -4  & 16 
  \end{array}
  \end{pmatrix}\xrightarrow[]{R_2=R_2-\frac{3}{4}R_1}
  \begin{pmatrix}
  \begin{array}{rr|r}
  4 & 3 & 5\\
  0 & \frac{-25}{4} & \frac{49}{4}
  \end{array}
  \end{pmatrix}\\\xrightarrow[]{R_2=\frac{-4}{25}}
  \begin{pmatrix}
  \begin{array}{rr|r}
  4 & 3 & 5\\
  0 & 1 & \frac{-49}{25}
  \end{array}
  \end{pmatrix}\xrightarrow[]{R_1=\frac{1}{4}R_1}
  \begin{pmatrix}
  \begin{array}{rr|r}
  1 & \frac{3}{4} & \frac{5}{4}\\
  0 & 1 & \frac{-49}{25}
  \end{array}
  \end{pmatrix}\\\xrightarrow[]{R_1=R_1-\frac{3}{4}R_2}
  \begin{pmatrix}
  \begin{array}{rr|r}
  1 & 0 & \frac{68}{25}\\
  0 & 1 & \frac{-49}{25}  
  \end{array}
  \end{pmatrix}        
\end{align}
Hence,
\begin{align}
\vec{A}=\myvec{\frac{68}{25}\\\frac{-49}{25}}
\end{align}
Thus, $\vec{A}(68/25,-49/25)$ is the coordinate from Point $\vec{P}$ to the line $3x-4y-16=0$.
\begin{figure}[!h]
	\begin{center} 
	    \includegraphics[width=\columnwidth]{figs/lines.png}
	\end{center}
\caption{Foot of Perpendicular from point P and given line}
\label{fig:Fig}
\end{figure}
\end{enumerate}
\end{document}